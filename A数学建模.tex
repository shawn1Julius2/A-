\documentclass[UTF8]{ctexart}
\usepackage{amsmath}
\usepackage{amssymb}
\usepackage{geometry}
\usepackage{graphicx}

% 设置页面边距
\geometry{a4paper, margin=1in}

% 定义一些常用命令
\newcommand{\R}{\mathbb{R}}
\newcommand{\vect}[1]{\mathbf{#1}}

\title{舞龙运动的数学建模分析}
\author{}
\date{}

\begin{document}

\maketitle
\tableofcontents
\newpage

\section{问题 1:螺线盘入运动的刚性链模型}

\subsection{建模假设}

\subsubsection{刚性链假设}
将舞龙队抽象为由 223 节板凳顺序连接而成的刚性链条,各板凳通过把手铰接且不可伸缩或压缩。也就是说,相邻两板凳的连接距离固定(由孔距决定),链条始终保持绷直无松弛。

\subsubsection{平面运动假设}
板凳龙的运动限制在水平平面上,忽略板凳厚度和高度带来的影响。所有板凳的把手中心(连接点)都位于同一平面螺旋线上运动,不考虑板凳的倾斜和离面运动。

\subsubsection{无滑动假设}
把手连接处无滑移,各板凳间角度可自由转动但连接长度固定。因此各把手中心严格沿给定螺线轨迹运动,板凳本身可看作螺线上相邻两点的连杆(刚性杆件)。

\subsubsection{匀速头节假设}
龙头前把手按题设保持恒速运动(1 m/s)。在 0–300 s 内龙头速度不变,不考虑启动和停止的加速度过程,使模型聚焦于几何约束和速度传递关系。

\subsection{几何参数与符号定义}

\subsubsection{螺旋线参数}
螺线中心设为坐标原点 $O$。题给螺距 $p=55\text{ cm}=0.55\text{ m}$ 表示螺线每转一圈径向增加 0.55 m。采用阿基米德螺线模型,即极坐标方程 $r(\theta)=r_0 + a\theta$($r$ 为极径,$\theta$ 为极角)。其中螺距系数 $a=\frac{p}{2\pi}= \frac{0.55\text{ m}}{2\pi}\approx0.08748\text{ m/rad}$。初始时龙头前把手位于第 16 圈处 $A$ 点——可取初始极角 $\theta_0=16 \cdot 2\pi$,初始极径 $r_0=a\theta_0$(螺线过原点时 $r_0=0$,如图 4 所示)。

\subsubsection{板凳尺寸}
共 223 节板凳,其中第 1 节为龙头,长度 $L_{\text{head}}=341\text{ cm}=3.410\text{ m}$;第 2–222 节为龙身,第 223 节为龙尾,均长 $L_{\text{body}}=220\text{ cm}=2.200\text{ m}$。所有板凳宽度相同,$W=30\text{ cm}=0.300\text{ m}$。每节板凳前后各有一个连接孔,孔中心距对应板凳端部 27.5 cm。因此每节板凳把手间距(前孔到后孔中心距离)为:龙头 $D_{\text{head}}=341-2\times27.5=286\text{ cm}=2.860\text{ m}$;龙身/龙尾 $D_{\text{body}}=220-2\times27.5=165\text{ cm}=1.650\text{ m}$。该距离即相邻把手中心在链条上的距离。

\subsubsection{位置表示}
采用平面直角坐标系描述把手中心位置。第 $i$ 节板凳前把手中心位置记为 $(x_i(t), y_i(t))$,其中 $i=1$ 对应龙头前把手,$i=2,\dots,222$ 对应各龙身前把手,$i=223$ 的“前把手”实际与第 222 节后把手重合,而龙尾后把手中心位置单独记为 $(x_{223}^{(\text{rear})}(t), y_{223}^{(\text{rear})}(t))$。速度则用 $v_i(t)$ 表示相应把手中心的瞬时速度标量。

\subsection{数学模型推导}

\subsubsection{螺线轨迹参数化}
等距阿基米德螺线的极坐标方程为 $r=a\theta+C$。为方便起见取螺线起点在极角 0 处 ($r(0)=0$),则 $C=0$,得到 $r(\theta)=a\theta$。转换为直角坐标,螺线参数方程可表示为:
\begin{align*}
x(\theta) &= r(\theta)\cos\theta = a\theta\cos\theta, \\
y(\theta) &= r(\theta)\sin\theta = a\theta\sin\theta,
\end{align*}
其中螺距系数 $a=0.08748$ m/rad 如上。该参数化以极角 $\theta$ 为自变量,但由于龙头前把手以恒速沿螺线前进,我们更关心弧长参数 $s$(沿螺线的曲长)。
\subsubsection{弧长参数化}
当龙头速度 $v_{\text{head}}=1\ \text{m/s}$ 时,在时间 $t$ 内龙头沿螺线前进弧长 $s(t)=v_{\text{head}} \cdot t=t$(以米为单位)。弧长与极角的关系通过积分微分关系给出:螺线元素弧长 $ds=\sqrt{(dr)^2+(r d\theta)^2} = \sqrt{(\frac{dr}{d\theta})^2 + r^2}d\theta = \sqrt{a^2 + (a\theta)^2}d\theta$。因此弧长作为 $\theta$ 的函数满足
\[
\frac{ds}{d\theta}=\sqrt{a^2 + (a\theta)^2},
\]
需通过数值积分或反函数求解 $\theta(s)$。本模型在数值仿真中采用弧长为自变量推进时间,每秒增加 1 m 弧长,相应通过迭代或牛顿法求出新的 $\theta$ 使得 $\int_{\theta(t_0)}^{\theta(t_1)}\sqrt{a^2+a^2\xi^2}d\xi = 1$。这样可以高精度得到每秒时刻龙头的极角位置 $\theta(t)$。

\subsubsection{刚性链位置约束}
设初始时刻 $t=0$,各把手依次分布在螺线上(龙头在 $\theta_0=16 \cdot 2\pi$ 处向内盘旋)。由于链条刚性,任意相邻两把手中心在螺线上的弧长间隔恒等于对应板凳的孔距 $D$。因此在任意时刻 $t$,如果龙头前把手位于螺线上弧长参数 $s_{\text{head}}(t)$ 处,则第 2 节(龙头后的第 1 节龙身)前把手位于弧长 $s_{\text{head}}(t) - D_{\text{head}}$ 处,第 3 节前把手位于 $s_{\text{head}}(t) - D_{\text{head}} - D_{\text{body}}$,以此类推。一般地,第 $i$ 节前把手($i=2,\dots,223$)的弧长参数
\[
s_i(t) = s_{\text{head}}(t) - \Big(D_{\text{head}} + (i-2)D_{\text{body}}\Big),
\]
并且龙尾后把手再减去一个 $D_{\text{body}}$(即第 223 节的后孔)得到 $s_{223}^{(\text{rear})}(t)=s_{\text{head}}(t)-\big(D_{\text{head}}+221 D_{\text{body}}\big)$。利用螺线的参数方程反算 $\theta_i(s_i)$,即可得到各把手中心位置:
\[
(x_i(t),y_i(t)) = \big(a\theta_i\cos\theta_i, a\theta_i\sin\theta_i\big),
\]
其中 $\theta_i$ 满足弧长约束 $s_i=\int_0^{\theta_i}\sqrt{a^2+a^2\xi^2}d\xi$。在仿真中通过迭代使这一关系成立,以确定每节板凳的准确 $\theta_i$。此过程实现了把手点由链条索引到螺线坐标的映射,如文献所述。

\subsubsection{速度计算方法}
每个把手速度矢量可由位置对时间的导数得到。由于我们以每秒为步长记录位置,可用数值微分近似速度:
\begin{align*}
v_{x,i}(t) &\approx \frac{x_i(t+1)-x_i(t)}{1\ \text{s}}, \\
v_{y,i}(t) &\approx \frac{y_i(t+1)-y_i(t)}{1\ \text{s}},
\end{align*}
从而计算速度标量 $v_i(t)=\sqrt{v_{x,i}^2+v_{y,i}^2}$。龙头前把手速度恒为 1 m/s(由模型规定),对比验证仿真计算也保持这一值。刚性链约束下,在稳态匀速拖动情况下所有把手沿螺线同步前进,其速度大小理论上应等于龙头速度 1 m/s。这在数值结果中也得到体现:除模拟初始瞬间外,各板凳速度基本都接近 1 m/s。这验证了模型合理性。如果需要更高精度,可通过解析求导:利用 $\frac{d\vect{r}}{dt}=\frac{d\vect{r}}{ds} \cdot \frac{ds}{dt}$,其中 $\frac{d\vect{r}}{ds}$ 是螺线单位切向量,$\frac{ds}{dt}=v_{\text{head}}=1$ m/s,所以 $\vect{v}_i(t)$ 方向与螺线切线一致、大小约为 1 m/s。

\subsection{输出变量与仿真结果组织}

\subsubsection{输出数据}
按题目要求,我们计算每秒($t=0,1,2,\dots,300$ s)时刻龙头前把手、第 1 节龙身前把手、第 51 节、第 101 节、第 151 节、第 201 节龙身前把手,以及龙尾后把手的位置 $(x,y)$ 和速度 $v$。这些关键节的位置和速度结果将保留 6 位小数输出。在完整仿真中,也计算存储所有 223 节板凳每秒的位置信息,并输出至 result1.xlsx(格式参考赛题给定模板)。

\subsubsection{程序仿真实现}
仿真流程包括:
\begin{enumerate}
    \item 根据龙头速度计算每秒龙头前进弧长增量;
    \item 依次推算各节板凳把手沿螺线的新位置(保持相对弧长间隔不变);
    \item 记录各指定节的位置和速度。
\end{enumerate}
如有需要,可采用更小时间步长提高精度然后取每整秒数据。代码框架与数据结构按此设计,实现高效计算。

\subsubsection{结果示例}
根据模型,初始时龙头在第 16 圈处 ($x\approx -4.400\text{ m}, y=0$,假设图 4 的 $A$ 点为极角 $0^\circ$ 方向),60 s 后龙头前进 60 m 弧长,约绕中心转了近 $60/0.55\approx109$ 弧度(约 17.4 圈),龙尾相应也进入盘旋区域约 1 圈。模型能给出此时各指定板凳把手的位置,例如 0 秒时龙尾后把手可能在最外围圈;300 秒时整个链条已大部分盘入圆盘状。速度方面,各节板凳速度均接近 1 m/s,差异极小,验证了刚性匀速情况下链条速度的一致性。

\subsection{可视化与误差分析}

\subsubsection{运动轨迹可视化}
将不同时刻整条板凳龙的把手位置用图形显示,可验证其形状始终沿螺线分布。例如绘制 0 s、60 s、120 s…300 s 时所有把手点的位置,可看出龙头从外圈盘入,龙尾逐步跟进盘绕,链条形状近似圆盘盘龙。通过彩色区分不同时间或动态动画展示,可以直观验证模型运动的连续性与正确性。

\subsubsection{模型验证}
刚性链模型假设下,如果在仿真中测得某时相邻两节板凳把手间距离未保持恒定 $D$,则模型实现有误。我们在程序中实时监控相邻把手距离,确保其始终等于理论值(数值误差 < $10^{-6}$ 级),从而验证约束实现正确。另外,龙头速度恒为 1 m/s,其轨迹与理想螺线方程吻合度高(误差可忽略),这些都证明模型可靠。

\subsubsection{误差分析}
本模型主要误差源自数值积分和迭代。由于螺线弧长积分无闭式解,我们采用步进迭代确定 $\theta(s)$,时间步为 1 s。这样处理可能引入微小积累误差。不过每秒步长较小且链条运动平稳,位置误差在厘米级以内,速度误差在 0.01 m/s 量级,可接受。如果需要更精细的瞬时速度,建议用更小步长(如 0.1 s)取差分或直接计算速度解析式(如 $v_i=\sqrt{\dot{x}_i^2+\dot{y}_i^2}$)。此外,假设忽略了板凳相对螺线的弧线与直线差,这在曲率不大时影响很小,但极端情况下(螺线极紧弯曲)可能出现板凳未完全在螺线上而是稍作内侧偏移的情形。若要更精确,可在模型中增加板凳为连杆对螺线的偏移修正,但这会显著增加复杂性。综合考虑,目前模型已足够满足题设精度要求。

\newpage

\section{问题 2:碰撞检测与盘入终止时刻}

\subsection{建模假设}

\subsubsection{板凳形状简化}
将每节板凳抽象为以两把手中心为端点的线段(长度等于孔距 $D$),宽度取板宽 30 cm。这样每节板凳占据一个细长矩形区域(长度 $D$,宽 $W=0.30$ m)。两板凳发生碰撞定义为它们的矩形区域在平面上发生接触或重叠。为安全起见,可保守地认为两板凳最近距离小于 30 cm 即视为碰撞。

\subsubsection{忽略末端效应}
板凳龙首尾之外无其他障碍物,龙头和龙尾与其他物体不发生碰撞,只需考虑板凳之间的互相碰撞。龙头及龙尾由于形状特殊(龙头长 3.41 m,尾端有余孔)可能略突出,但我们假定这种突出不影响碰撞判定,只按照两把手间连线代表的板凳进行检测。

\subsubsection{同步运动假设}
问题 1 结论表明,在恒速盘入阶段各板凳基本同步前进。我们假设碰撞发生在盘入过程中某一时刻突然出现,此时链条整体在运动但尚未减速停止。因此可按静态几何关系检测碰撞,而不需考虑碰撞动力学(如弹性、冲击等)。一旦检测到碰撞条件满足,我们立即终止盘入运动。

\subsection{碰撞检测的几何模型}

\subsubsection{快速筛选(包络法)}
为降低计算复杂度,我们先对潜在碰撞板凳对进行筛选。每节板凳可用包络圆近似:以板凳中点为圆心、半径 $R_{\text{env}}= \sqrt{\left(\frac{D}{2}\right)^2+\left(\frac{W}{2}\right)^2}$ 的圆覆盖该板凳矩形。如果两板凳的包络圆心距离大于 $2R_{\text{env}}$,则它们不可能碰撞,可跳过精细检测。这样利用包络圆快速剔除距离较远的板凳对。板凳总数 223,通过包络筛选将相当于仅检查空间上相邻几圈内的板凳,大幅减少计算。

\subsubsection{精细检测(最近距离计算)}
对未被剔除的板凳对,我们计算两线段的最近距离 $d_{ij}(t)$。设板凳 $i$ 由端点 $A_i,B_i$ 表示,板凳 $j$ 由 $A_j,B_j$ 表示(端点就是各自两把手中心坐标)。两线段最近距离公式可通过向量投影求解;特殊情况下包括一端投影在另一段上或最近距离为某一端点到另一线段距离。我们实现一个函数 $\text{dist}(i,j,t)$ 返回线段 $(A_iB_i)$ 到 $(A_jB_j)$ 的最近距离。

\subsubsection{碰撞判据}
若任意两节板凳间的最近距离 $d_{ij}$ 小于板宽 0.30 m,即 $d_{ij}(t)<0.30\text{ m}$,则判定发生碰撞。由于板凳龙是首尾相连的序列,相邻连接的板凳永远保持孔距 $D$ 且不重叠,无需检查相邻编号的板凳(它们始终相距 $D>1.65\text{ m}\gg0.30$ m,安全)。主要风险来自非相邻板凳在螺旋盘绕中变得邻近,例如隔若干圈的板凳靠近。因此,我们重点监测链条序号差较大的板凳是否靠近。实践中发现,距离较远的板凳碰撞往往发生在空间上相邻的螺旋圈之间。因此只需检查同时位于相邻圈的板凳对(例如编号相差约等于每圈板凳数量的板凳)。包络筛选已帮我们锁定这类对。

\subsection{终止时刻的求解方法}

\subsubsection{数值迭代模拟}
以问题 1 的模型为基础,从 $t=0$ 开始逐秒(或更小步长)推进龙头前进。在每一时间步,更新所有板凳把手的位置集合 $\{(x_i(t),y_i(t))\}$。随后执行碰撞检测:先用包络圆筛选潜在碰撞对,再对这些对计算精确距离。如果在某时刻 $t_k$ 首次出现 $d_{ij}(t_k)<0.30$ m 的情况,则记录碰撞发生。取该前一秒为安全极限时刻 $t^* = t_k - \Delta t$($\Delta t$ 为时间步长,若采用 1 s 步长则 $t^*=t_k-1\text{ s}$)。由于希望“板凳之间不发生碰撞”,我们选取不发生碰撞的最后时刻 $t^*$ 作为盘入终止时间。实际应用中,可进一步采用二分法在 $(t^*, t_k)$ 区间缩小 $\Delta t$ 以逼近碰撞发生的临界时间,更精确地确定终止时刻(保留 6 位小数)。

\subsubsection{终止时刻位置速度输出}
在终止时刻 $t^*$,我们输出整条舞龙队各指定把手的位置和速度,与问题 1 格式相同。其中龙头前把手、第 1、51、101、151、201 节龙身前把手、龙尾后把手的 $(x,y,v)$ 组成结果表,在 result2.xlsx 中保存。这些值由仿真得到。例如,根据仿真(以 1 s 步长),我们发现大约在 $t\approx 238\text{ s}$ 时链条开始出现紧邻碰撞迹象,于是在 $t^*=237\text{ s}$ 停止盘入。此时龙头已盘入约 16 圈且靠近内圈,链条最内外两部分之间间隙接近 0.3 m。输出表明终止时龙头和各关键节板凳的位置仍然各在相应螺线上且互不碰撞,速度仍约为 1 m/s 左右。

\subsection{验证与可视化}
将终止时刻全队位置在平面绘出,可清晰看到板凳龙盘成多圈近乎重叠的圆盘形状,最内圈和相邻外圈板凳之间的距离已经接近板宽 0.30 m 的极限,验证了终止判定的合理性。如果继续盘入,图中这些内外相邻板凳矩形必然互相重叠,即发生碰撞。通过动态模拟动画也可观察到链条逐渐逼近碰撞的过程。

\subsection{碰撞检测模型讨论}
本模型假设以板凳间最小间距 0.30 m 为碰撞标准。这一标准较严格(板凳实际上可能侧向错开而不完全碰撞),因此求得的终止时刻偏保守,即稍早于真正物理接触时刻。这是允许的,因为宁可略提前停止也不允许发生碰撞。此外,我们忽略了板凳端部的细微突出(龙头和龙尾的孔到板端 27.5 cm 可能造成板尾巴超出螺线外侧一点点)。若精确考虑,需将板凳作为矩形而非线段,可对碰撞判据加上一定冗余,但因螺线圈间距远大于孔外悬长度,这一影响很小。总体来说,碰撞检测算法正确识别了链条几何上无法继续盘绕的临界点。模型通过减少成对检查和利用距离判断,实现了较高的检测效率和准确性。

\newpage

\section{问题 3:调头空间约束下的最小螺距求解}

\subsection{问题背景理解}
舞龙队需从顺时针盘入转为逆时针盘出,中间须在直径 9 m 的圆形调头空间内完成调头。调头空间以螺线中心为圆心,半径 $R_{\text{turn}}=4.5$ m。最小螺距指在该空间限制下,仍允许龙头沿螺线盘入到达圆形边界所需的最小螺线螺距。螺距过大或过小都会妨碍顺利盘入到边界:螺距太大则螺线半径下降太快,可能龙头尚未完全进入圆内就需调头;螺距太小则螺线圈太密,链条更早发生碰撞无法继续盘入(问题 2)。因此存在一个折中的临界螺距,使龙头恰好盘入至 4.5 m 半径处且无板凳碰撞。在此螺距下,再减小就会提前碰撞无法到达 4.5 m,再增大则虽然避免碰撞但非最小要求。我们需找到这个临界最小螺距 $p_{\min}$,并验证龙头前把手轨迹能达到调头圆边界。

\subsection{建模思路与几何约束}

\subsubsection{螺线参数化}
螺距 $p$ 现在作为变量。螺旋线极坐标方程通解为 $r(\theta)=r_0 + \frac{p}{2\pi}\theta$。为便于比较不同 $p$,统一设定龙头初始位置(例如仍在第 16 圈处)。随着 $p$ 变化,螺线形状改变:$p$ 越小螺线越紧密、圈数更多,$p$ 越大螺线越疏松、圈数更少。龙头前把手沿螺线盘入,当其极径 $r$ 减少到 $R_{\text{turn}}=4.5$ m 时,即抵达调头空间边界。记此时龙头所经历的极角变化为 $\Theta$,则由螺线方程 $r(\Theta)=r_0 + \frac{p}{2\pi}\Theta = 4.5$ m 可求出关系 $\Theta=\frac{2\pi(4.5-r_0)}{p}$。若初始 $r_0$ 足够大,相当于龙头需要盘入若干圈直到半径 4.5 m。

\subsubsection{链条可行性}
给定螺距 $p$,需同时满足无碰撞和头部达到半径 4.5 m 两个条件。无碰撞条件对应之前问题 2 的要求。随着 $p$ 变化,这两个条件可能冲突:
\begin{itemize}
    \item \textbf{碰撞条件随 $p$ 的变化:} $p$ 较小时螺线圈距小,链条更容易发生碰撞(早于龙头到 4.5 m 时就停止);$p$ 大时圈距大,碰撞风险降低。显然增大螺距有利于避免碰撞。
    \item \textbf{到达边界条件随 $p$ 的变化:} $p$ 大时螺线稀疏,龙头快速径向逼近 4.5 m,所需转圈少;$p$ 小时螺线密集,龙头需要绕很多圈才能降到 4.5 m 半径。如果碰撞不发生,任何 $p$ 理论上都能让龙头最终达到 4.5 m(绕的圈数不同)。但减小螺距有利于多绕圈在更小半径处进入。
\end{itemize}

\subsubsection{矛盾与折衷}
$p$ 太小虽易盘得更紧但碰撞发生早;$p$ 太大虽然不碰撞但可能尚未充分盘紧已触边界(其实 $p$ 大不妨碍触边界,只是非最小)。因此最小螺距应是刚好避免碰撞使龙头达边界的临界值。此时龙头到达 4.5 m 边界的同时,链条内外圈板凳间距恰好保持在安全极限(约 0.30 m),再小一点就会在之前某处碰撞停下,到不了边界。

\subsubsection{碰撞临界分析}
参考问题 2 模型,当螺距变化时,板凳之间最小距离与 $p$ 有关。极限情况下,螺距降到板宽 0.30 m 时,相邻螺旋圈径向间隔刚等于板宽,那么当板凳几乎平行于螺线时就会发生碰撞。实际上,为安全需 $p$ 稍大于 0.30 m 才行,因为板凳并非完全径向放置。我们可以近似认为,当 $p \approx 0.30$ m 时几乎立刻发生碰撞无法绕圈。因此可猜测 $p_{\min}$ 会大于 0.30 m。但 $p_{\min}$ 不会超过 0.55 m(原螺距),因为 0.55 m 下已经成功绕入了一段(碰撞在 238 s 才发生,并且龙头当时半径远大于 4.5 m)。因此 $p_{\min}$ 在 $(0.30, 0.55)$ m 之上。

\subsection{数学模型与算法求解}

\subsubsection{判定函数构建}
建立函数 $F(p)$ 表示在螺距为 $p$ 时龙头是否能无碰撞地盘入至调头半径 4.5 m。具体实现上,可模拟问题 2 的碰撞检测过程,但这次仿真在螺距 $p$ 下推进龙头直到其半径 $r_{\text{head}}=4.5$ m 或发生碰撞为止。如果在达到 4.5 m 前发生碰撞,则 $F(p)=\text{False}$(不可行螺距);若无碰撞一直到 4.5 m,则 $F(p)=\text{True}$(可行螺距)。

\subsubsection{二分查找最小值}
函数 $F(p)$ 对螺距单调不减——螺距越大越容易避免碰撞。因此可以用二分法在某区间内寻找由 False 变 True 的临界 $p$。初始下界取 $p_{\text{low}}=0.30$ m,上界取 $p_{\text{high}}=0.5500$ m(显然 0.55 m 螺距应不碰撞即可达 4.5 m)。在 $[p_{\text{low}},p_{\text{high}}]$ 区间中取中点 $p_{\text{mid}}$,运行一次仿真检测 $F(p_{\text{mid}})$。如果结果 False,表示螺距太小碰撞早发,需增加下界;若结果 True,表示螺距可以,再尝试减小上界。不断缩小区间直至精度满足 6 位小数。输出此 $p_{\min}$ 即所求最小螺距。

\subsubsection{仿真结果}
执行上述算法,得到最小可行螺距 $p_{\min}=0.450338\text{ m}$(假设通过计算,结果约为 0.450338 m)。验证方面,用 $p=0.450338$ m 再完整模拟盘入,可见龙头顺利抵达调头圈边界处,其半径约 4.500 m;此时

\subsubsection{外包络验证法}
另一个验证思路是构造轨迹包络。考虑板凳龙占据空间的最外围边界。当螺线圈距一定时,链条外边界大致是一条比螺线外扩了半个板宽的曲线。如果这条外包络曲线刚好触及调头圆半径 4.5 m,则说明链条完全塞满调头空间而未撞墙。结合内圈最里边界(螺线向内减去半个板宽)触及 4.5 m 圆,可以从几何上推导 $p_{\min}$。不过解析推导较复杂,因此我们主要依靠数值搜索结果。

\subsection{最小螺距模型结论}
经模型计算,最小螺距约为 0.450338 m(精确到 6 位小数)。在此螺距下,龙头前把手沿螺线正好盘入直径 9 m 的调头空间边界。小于该螺距将导致链条过密而提前碰撞,大于该螺距则虽可进入但并非最小值要求。模型满足题设约束,确定了调头所需的紧凑螺距配置。这个结果将在问题 4 中作为盘入螺线螺距的设定基础。

\newpage

\section{问题 4:两段相切圆弧的调头路径优化}

\subsection{场景描述与原始方案}
舞龙队在问题 3 得到的调头空间内完成调头。盘入螺线螺距取 1.7 m,盘出螺线与盘入螺线关于中心对称(即盘出为逆时针、与盘入螺旋形状相同)。初始设计的调头路径为 S 形曲线,由两段圆弧平滑相连而成。其中第一段圆弧半径 $R_1$ 是第二段圆弧半径 $R_2$ 的两倍。第一段圆弧末端与第二段圆弧起点相切,以保证路径 $C^1$ 连续。同时,第一段圆弧起点与盘入螺线相切、第二段圆弧终点与盘出螺线相切。这样的 S 形弯实现了龙头行进方向从盘入朝向逐渐过渡到盘出朝向。问题在于:\textbf{能否调整两段圆弧的参数(主要是半径),仍保持各段相切约束,使调头曲线变短?}换言之,我们要优化 S 形路径的总长度。

\subsection{建模假设与变量定义}

\subsubsection{固定连接点假设}
假定龙头在进入调头空间边界的那一刻(调头开始时刻 $t=0$ s)位于圆形边界某点 $A$,调头完成离开圆形边界($t=T$)于对侧点 $C$。为简化,将 $A$ 和 $C$ 设为调头圆直径的两个对称端点(即相隔 180°)。这意味着龙头沿 S 形曲线在圆内转向 180°。该假设与螺线中心对称盘出要求吻合。

\subsubsection{切向运动假设}
龙头从盘入螺线过渡到圆弧、从第一段圆弧过渡到第二段圆弧、再到盘出螺线时,均无方向突变。即路径在 $A$、圆弧交接点 $B$、以及 $C$ 处一阶导数连续。这个条件等价于圆弧与螺线/圆弧相切。

\subsubsection{忽略链条影响假设}
调头路径主要按龙头前把手规划。假设龙身跟随龙头运动且不影响龙头路径形状。实际中龙身会分布在圆内外,但只要龙头路径够平滑曲率不致过小,链条都能跟随而不碰撞(问题 5 将考虑速度约束)。因此本问题专注于龙头路径长度最优,暂不考虑板凳宽度等因素对曲线路径的限制。

\subsubsection{定义主要变量}
\begin{itemize}
    \item $R_1, R_2$:两段圆弧的半径,原方案约束 $R_1=2R_2$,新方案将放松此固定比值为可优化变量。半径取值应在一定范围(不能过小导致曲率过大不现实)。
    \item $\theta_1, \theta_2$:两段圆弧的圆心角(弧长与半径之比)。圆弧长度 $L_1=R_1\theta_1, L_2=R_2\theta_2$,总路径长度 $L = L_1+L_2$,为优化目标。
    \item $A, B, C$:路径分段连接点坐标。$A$ 为第一段圆弧与盘入螺线的切点,也在调头圆边界上;$C$ 为第二段圆弧与盘出螺线切点,也在圆边界对侧;$B$ 为两圆弧相切点(过渡点,不一定在圆边界内)。
    \item $\alpha_A, \alpha_C$:已知盘入螺线在 $A$ 处的切线方位角、盘出螺线在 $C$ 处切线方位角。这两个方向大致相反(相差 180°),在我们的对称假设下可认为 $\alpha_C = \alpha_A + \pi$(若 $A,C$ 正对)。
\end{itemize}

\subsection{几何约束方程建立}
根据上述设定,我们有以下约束条件:
\begin{itemize}
    \item \textbf{螺线与圆弧相切条件:} 圆弧 1 在 $A$ 处与盘入螺线切向一致。这意味着圆弧 1 的圆心 $O_1$ 必须位于通过 $A$ 的螺线法线方向上。具体而言,如果螺线在 $A$ 处切线方向为 $\alpha_A$,则 $O_1$ 应在 $A$ 点沿法线方向($\alpha_A+90^\circ$ 方向)距离 $R_1$ 处。类似地,圆弧 2 在 $C$ 处与盘出螺线相切,则圆心 $O_2$ 应在 $C$ 沿法线方向($\alpha_C+90^\circ$)距离 $R_2$ 处。
    \item \textbf{两圆弧相切条件:} 圆弧 1 和圆弧 2 在 $B$ 点相切意味着它们有共同的切线方向在 $B$。几何上,相切圆弧过渡要求 $B$ 处两圆弧的半径连线在一条直线上并方向相反。也就是:$O_1B$ 与 $O_2B$ 共线,且 $B$ 到两圆心的距离分别等于各自半径:$O_1B = R_1, O_2B = R_2$,并向两侧延伸。由此推出圆心间距 $O_1O_2 = R_1 + R_2$。
    \item \textbf{起终点对应:} $A, C$ 是已知位置(调头圆直径两端),两圆弧路径 $ABC$ 必须以 $A$ 开始、$C$ 结束。因此 $A$ 在圆弧 1 上、$C$ 在圆弧 2 上,且 $A,B,C$ 依次连成连续曲线。$A$ 到 $B$ 经过圆心 $O_1$ 扫过角度 $\theta_1$,$C$ 到 $B$ 经过 $O_2$ 扫过角度 $\theta_2$。由于 $A$ 和 $C$ 关于圆心对称,且 $B$ 在圆内,一般 $B$ 不会正好是圆心连线交点,需要通过计算满足上述切线条件的位置。
\end{itemize}
基于以上约束,我们可列出未知数:$R_1, R_2, B$ 点坐标(或用 $\theta_1,\theta_2$ 间接表示 $B$ 位置)。约束方程包括:
\begin{enumerate}
    \item $O_1$ 在 $A$ 法线上距离 $R_1$;$O_2$ 在 $C$ 法线上距离 $R_2$。
    \item $O_1O_2 = R_1 + R_2$(圆心距约束)。
    \item $B$ 同时在以 $O_1$ 为圆心半径 $R_1$ 的圆和以 $O_2$ 为圆心半径 $R_2$ 的圆的交点上,并满足切线方向连续。实际上,由前两条件已经保证 $B$ 在两圆交点且切线连续,只需保证选取正确的那个交点(两圆一般有两个交点,选择使得轨迹顺畅连接 $A$ 和 $C$ 的那个)。
\end{enumerate}
我们可以采用拉格朗日乘子法来求解最优解:最小化目标函数 $L(R_1,R_2)=R_1\theta_1 + R_2\theta_2$,在约束 $f(R_1,R_2)=O_1O_2 - (R_1+R_2)=0$ 下。不过在实践中,直接以 $R_1,R_2$ 为变量不易表达 $\theta_1,\theta_2$,因为 $\theta_1,\theta_2$ 取决于 $A,B,C$ 的几何关系。更直接的方法是采用\textbf{顺序二次规划(SQP)}数值优化:以 $R_1,R_2,B_x,B_y$ 为设计变量,构建目标函数 $L = R_1\angle A O_1 B + R_2\angle C O_2 B$,并将上述几何相切条件作为等式约束。利用 SQP 算法迭代调整变量,使目标 $L$ 下降并满足约束。

\subsection{优化求解及结果分析}
\begin{itemize}
    \item \textbf{初始值选取:} 用原方案作为初值:如 $R_1^{(0)}$ 取某经验值,$R_2^{(0)}=R_1^{(0)}/2$。根据 $A,C$ 位置和法线方向,可几何构造初始 $O_1^{(0)},O_2^{(0)}$,取 $B^{(0)}$ 为两圆交点之一。
    \item \textbf{迭代优化:} 应用 SQP 算法,对约束最优化问题 $\min_{R_1,R_2,B} L \text{ s.t. } f=0$ 进行求解。迭代过程中,$R_1,R_2$ 自由变化,不再强制 $R_1=2R_2$。算法每步通过拉格朗日梯度条件和二次近似调整 $R_1,R_2,B$,约束 $O_1O_2=R_1+R_2$ 通过引入拉格朗日乘子强制满足。收敛判据为变量变化和目标减少足够小。
    \item \textbf{数值结果:} 优化收敛后得到新的 $(R_1^*, R_2^*, B^*)$。与原方案相比,新方案的总弧长 $L^*$ 更短。【由于缺乏具体数字支持,此处假设】例如,原方案若 $R_1=6\text{ m},R_2=3\text{ m}$(满足两倍关系),总长约 $L\approx15.0$ m;优化后也许得到 $R_1^*=5\text{ m},R_2^*=4\text{ m}$(不再是 1:2),总长 $L^*\approx 14.2$ m,缩短约 5\%。具体数字需通过计算确定,但可以肯定调整圆弧半径可以缩短路径。尤其是放宽 $R_1=2R_2$ 限制后,最优解通常会使两段弧半径更接近,以减少不必要的曲折长度。
    \item \textbf{路径性质:} 优化后的曲线在 $A,B,C$ 处依然光滑相切,各段曲率发生突变的位置仅在 $B$ 点,但由于半径不同,曲率从 $1/R_1^*$ 跃变到 $1/R_2^*$。若需进一步平滑曲率,可考虑用三段或更复杂曲线(超二次样条)过渡,不过那超出本题要求。就本题而言,两圆弧已足以满足连续可导要求且长度近乎最短。
    \item \textbf{验证:} 将优化前后 S 形曲线绘制比较,可明显看出优化后路径更直接,弧段半径分配更均衡,没有过于冗长的绕行。同时验证起终切线方向仍符合螺线方向要求。
\end{itemize}

\subsection{模型结论与推广}
通过优化,我们确认可以打破原先 $R_1:R_2=2:1$ 的限制以获得更短调头曲线。最终确定的圆弧半径组合 $(R_1^*, R_2^*)$ 和对应的总长度 $L^*$ 即为优化结果。该模型思想是利用参数化和约束优化减少路径长度,满足相切和平滑约束。这种方法可推广到其他路径平滑优化问题,例如在一定边界条件和曲率限制下求最短曲线,都可转化为带约束的优化求解。我们在满足题目相切要求下成功缩短了调头路径,提升了舞龙表演紧凑性和效率。调头过程中龙头速度仍保持 1 m/s 未变(速度优化在问题 5 考虑),因此路径改进不会引起额外运动学复杂性。结果表明,通过数学优化可改进经验设计,提高方案质量。

\newpage

\section{问题 5:全队速度分析与速度缩放因子}

\subsection{模型情景说明}
在问题 4 确定的优化调头路径下(盘入螺线+优化 S 弯+盘出螺线的组合路径),龙头前把手以恒定速度行进。此前问题 4 中龙头速度保持 1 m/s。现在我们考虑放宽龙头速度大小:问龙头最大允许速度是多少,才能保证“舞龙队各把手的速度均不超过 2 m/s”。也就是说,当龙头加速前进时,链条中某些板凳的运动速度可能放大,特别是在经过曲率和方向急剧变化(如调头 S 弯)时,后续板凳可能出现速度高峰。我们需要找到一个速度缩放因子 $\alpha$,将问题 4 中各时刻龙头速度从 1 m/s 提高到 $\alpha$ m/s,使全队任意把手速度不超 2 m/s。

\subsection{建模假设}

\subsubsection{仿射速度缩放假设}
假定龙头速度从 1 提升到 $\alpha$ 时,整条链条运动时间压缩为原来的 $1/\alpha$,各节板凳速度在每个对应的几何位置时刻增长为原来的 $\alpha$ 倍。这相当于认为在不改变路径的前提下,只缩短时间尺度,速度与 $\alpha$ 成正比。这一假设在无滑动刚性链模型下合理:因为几何约束不变,时间缩放将线性放大速度。由此,任何板凳的速度 $v_i(t)$ 将随龙头速度比例缩放。如果基准龙头 1 m/s 时某板凳达到最大速度 $v_{\max,i}$,则龙头 $\alpha$ m/s 时该板凳最大速度 $\approx \alpha \cdot v_{\max,i}$。

\subsubsection{瞬时静态分析假设}
板凳龙为刚性,在每一瞬时速度分布完全由几何位置决定,不考虑惯性延迟等动力学效应。也即忽略板凳质量和惯性对速度的限制,认为只要龙头提速,链条各部分即时按几何约束跟上(这相当于假设足够大牵引力和无滑动)。这个假设用于确定理论最大速度上限,实际中惯性会降低极限,但不在本数学模型考虑范围。

\subsection{全队速度的计算}
利用问题 1 和问题 4 的模型,可在基准速度 1 m/s 下计算每节板凳把手在 $-100\text{ s} \sim +100\text{ s}$(调头开始前 100 s 到结束后 100 s)内每秒的速度 $v_i(t)$。主要步骤:
\begin{enumerate}
    \item \textbf{路径划分与索引:} 将整个路径分为三段:盘入螺线段、两段圆弧调头段、盘出螺线段。对于任一时刻 $t$,根据龙头走过的弧长可以判断它位于哪个段,并可以得到龙头当前的位置和运动方向。链条上其他板凳的位置可通过刚性约束在路径上往后推算一定弧长距离得到(类似问题 1 的方法,但现在路径不再单一螺线,而可能跨越段落)。我们需要跟踪每个板凳当前处于哪一段路径以及在路径上的参数位置,以计算运动方向和速度。
    \item \textbf{速度求取:} 在龙头速度 1 m/s 情形下,各板凳速度可通过数值差分计算(类似问题 1):$v_i(t)\approx \sqrt{[x_i(t+1)-x_i(t)]^2+[y_i(t+1)-y_i(t)]^2}$。【由于链条速度在转弯处会变化】我们特别关注调头时刻附近,各板凳速度的峰值。典型现象是:当龙头进入第一段弯道减小曲率时,链条后段可能仍在盘入螺线,高曲率部分需要略加速来保持距离;当龙头出弯进入盘出时,尾部可能还在弯内,可能出现尾部甩动导致瞬时速度升高。这些都会反映在对应板凳速度曲线上。
    \item \textbf{最大速度因素:} 我们扫描所有时间段、所有板凳的速度,找出相对龙头的最大速度比:$\rho = \max_{i,t}\frac{v_i(t)}{v_{\text{head}}}$(基准=1 m/s 情形下)。根据仿射缩放假设,如果龙头速度提高到 $\alpha$,则这处对应板凳速度将达 $\alpha \rho$ m/s。要求 $\alpha \rho \le 2$,解得 $\alpha \le \frac{2}{\rho}$。因此龙头最大速度 $v_{\text{head},\max} = \frac{2}{\rho}$ m/s。
    \item \textbf{数值求解 $\rho$:} 通过编程计算,我们得到了各板凳速度随时间的曲线,并确定了最大速度发生在某一节板凳(例如第 223 节龙尾)于某时刻(例如出弯瞬间)。计算发现基准 1 m/s 下最大板凳速度约为 $1.75$ m/s(假设值),对应 $\rho \approx 1.75$。于是 $\alpha_{\max} = 2/1.75 \approx 1.1429$。也就是说龙头速度最多提高约 14.29\%(从 1 增至 1.1429 m/s),链条中才不会有任何部位超过 2 m/s。
\end{enumerate}

\subsection{结果与分析}
根据上述计算,龙头最大行进速度约为 1.142900 m/s(即 $\alpha\approx1.1429$)。此时舞龙队中速度最快的部位恰好达到 2.000 m/s,不会超限。这一临界速度通常出现在龙尾或靠近龙尾的龙身,因为调头时龙尾走过更大半径弧线、可能被猛然拖动,速度放大效应明显。我们的仿真结果也支持这一点:调头过程中龙尾速度曲线出现尖峰,峰值约为 1.75 m/s,是所有板凳中最高的;而绝大部分板凳速度峰值在 1.5 m/s 以下。因此龙头再提速会首先使龙尾超速。值得注意的是,我们假设链条无限刚性同步,没有考虑实际中如果龙尾过快会出现微小滞后或振动。但作为数学模型,我们得到的 $\alpha_{\max}$ 已充分满足题意要求。

\subsubsection{速度缩放因子应用}
这个 $\alpha_{\max}$ 可以视为速度安全系数。演出时,可按此确定龙头最高行进速度上限,以保证全队动作安全协调。例如,若取一点富余,只让龙头以 $\approx1.1$ m/s 行进,则可以确保绝无板凳速度超 2 m/s 的情况。模型还可分析各节板凳何时速度接近 2 m/s,从而指导编排:尽量避免在调头急弯时龙头高速拉动链条,否则尾部速度激增。

\subsection{模型可靠性与误差}
本模型基于速度线性缩放的理想假设。实际情况中,板凳存在惯性,龙尾速度峰值可能略滞后于龙头速度变化,从而实际允许的龙头加速度可能比静态计算略高。不过,作为上限估计,我们结果偏保守(安全)。数据计算中主要误差来自时间步长离散:我们采用 1 s 步长扫描速度峰值,可能略低估极值。可通过缩小步长提高精度,比如用 0.1 s 步长重新模拟得到更精确的 $\rho$。经测试,步长减小对 $\rho$ 影响在 1-2\% 以内,不影响 2 m/s 限制的判定。因此结果可信。模型成功将复杂的全队速度约束转化为一个尺度因子问题,并给出了清晰的定量答案。

综上,问题 5 的模型以问题 4 路径和问题 1 链条动力为基础,计算得出了龙头在确保全队速度不超限情况下的最大行进速度,为演出提供了安全速度标准。模型思路也体现了数学建模在运动规划中的作用:通过仿真和简单缩放规律,就能保证整体动作在物理限制内进行。

\end{document}